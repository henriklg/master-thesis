\documentclass[thesis.tex]{subfiles}

\begin{document}

% ----------------------------------------------------------
\chapter{Conclusion} \label{conclusion}
% ----------------------------------------------------------
% Present the results. Give critical discussion of your work and place it in the correct context. 
% ----------------------------------------------------------

\section{Results}
We have looked on some higly relevant papers written about automatic detection systems for medical videos from the last few years. From back when feature extraction methods consisted of selecting color and intensities thresholds, to newer and more sophisticated algorithms like CNN's have become mainstream. The newer methods may be more complex and harder to implement but we have found that these automatic feature extraction methods have a far higher accuracy and produce less false positives. We have also looked at the importance of having a big and varied dataset with labeled data. If the dataset is not large enough we can use several data augmentation methods to increase it, like the ones used in U-Net. 

\section{Further work}



\end{document}