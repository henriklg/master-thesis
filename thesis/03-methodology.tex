\documentclass[thesis.tex]{subfiles}

\begin{document}

% ----------------------------------------------------------
\chapter{Methodology} \label{methodology}
% ----------------------------------------------------------
% Present the method and system. Basically write everything you've done.
% ----------------------------------------------------------


% ----------------------------------------------------------
\section{Data collection} \label{datacollection}
% ----------------------------------------------------------
% Present how you gathered the data and labels
% ----------------------------------------------------------
The datasets used in our experiments are Kvasir Pillcam, Kvasir and Hyper-Kvasir. This section will demonstrate the main differences between the three datasets and explain how they can be found and used for fact checking. All three datasets are collected using endoscopic equipment at Vestre Viken Health Trust in Norway. The VV consists of 4 hospitals and provides health care for 470.000 people. One of the hospitals is Bærum Hospital, which has a large gastroenterology department from where the data is collected. 

\subsection{Kvasir PillCam} \label{kvasir_pillcam}
The dataset we used in our experiments consist of endoscopic videos collected from Bærum Hospital. Unlike Kvasir and Hyper Kvasir datasets we have made the Kvasir PillCam dataset for the purpose of this thesis. In total we have 44 videos which have gone through some re encoding to reduce the file sizes, and also because the original encoding is proprietary Sony technology. After that the videos are uploaded to Augeres tagging tool. The data export from Bærum also contains some findings for each video (if there is any) and are extracted, converted to frame number and that helped us a great deal with tagging the videos. We also have Thomas de Lange to thank, because he helped us a lot. When all 44 videos have been precisely labeled the dataset is exported from Augeres tagging tool and split into folders for each class. The folders/classes are given in table \ref{table:kvasir_pillcam}. In total we have 44 000 labeled images in 8 classes. In addition to labeling the images the dataset also contain a json file which stores coordinates for where in the frame the finding is located. The Kvasir Pillcam dataset will be an open-source dataset available for others scientists, and will later be grown to include more PillCam videos, both labeled and unlabeled samples.

\begin{table}
  \centering
  \begin{tabular}{ |c|c| }
  	\hline
  	Class number & Class name \\
    \hline
    0 & normal \\ 
    1 & polyp \\ 
    2 & polyrus \\ 
    \hline
  \end{tabular}
  \caption{PillCam class names and corresponding class numbers.}
  \label{table:kvasir_pillcam}
\end{table}


\subsection{Kvasir} \label{kvasir}
The Kvasir dataset \cite{KVASIRMultiClass17} contains images from inside the gastrointestinal (GI) tract. The samples are classified into three important anatomical landmarks and three clinically significant findings. In addition it has two classes related to the removal procedure of polyps. The dataset is sorted and annotated is performed by medical doctors. The class names and findings for each class is given in table \ref{table:kvasir}.

\begin{table}
  \centering
  \begin{tabular}{ |c|c|c| }
  	\hline
  	Class number & Class name & Number of samples \\
    \hline
    0 & normal & 8000 \\ 
    1 & polyp & 8000 \\ 
    2 & polyrus & 8000 \\ 
    \hline
  \end{tabular}
  \caption{Kvasir class names and corresponding class numbers.}
  \label{table:kvasir}
\end{table}


\subsection{Hyper Kvasir} \label{hyper_kvasir}



% ----------------------------------------------------------
\section{Data process} \label{data_pipeline}
% ----------------------------------------------------------
% Present everything related to working with data files
% ----------------------------------------------------------

\subsection{Data preprocessing}
% data normalization, splitting, augmentation etc

\subsection{Data pipeline}
% tensorflow.data.Dataset pipeline with prefecth, batching etc


% ----------------------------------------------------------
\section{System implementation} \label{systemimplementation}
% ----------------------------------------------------------


% ----------------------------------------------------------
\section{Summary} \label{03summary}
% ----------------------------------------------------------



\end{document}